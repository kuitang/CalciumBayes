\documentclass{article}
\usepackage{nips10submit_e,times}
%\documentstyle[nips07submit_09,times]{article}
\usepackage[square,numbers]{natbib}
\usepackage{amsmath, epsfig}
\usepackage{amsfonts}
\usepackage{subfigure}
\usepackage{graphicx}
\usepackage{amsfonts}
\usepackage{algorithm}
\usepackage{algorithmic}
\usepackage{easybmat}
\usepackage{footmisc}
\renewcommand\algorithmiccomment[1]{// \textit{#1}}
%
\newcommand{\ignore}[1]{}
\newcommand{\comment}[1]{}
\DeclareMathOperator*{\argmax}{arg\,max}

\title{Inferring Functional Connectivity of Neurons From Spiking Activity}

\author{
Shababo, B. \hspace{1cm} Stern-Asher, C.N. \hspace{1cm} Tang, K.\\
Columbia University, New York, NY 10027, USA \\
\texttt{bms2156@columbia.edu,conradsa@gmail.com,kt2384@columbia.edu}
%\texttt{pfau@neurotheory.columbia.edu} 
%\texttt{\{bartlett,fwood\}@stat.columbia.edu} 
}

% The \author macro works with any number of authors. There are two commands
% used to separate the names and addresses of multiple authors: \And and \AND.
%
% Using \And between authors leaves it to \LaTeX{} to determine where to break
% the lines. Using \AND forces a linebreak at that point. So, if \LaTeX{}
% puts 3 of 4 authors names on the first line, and the last on the second
% line, try using \AND instead of \And before the third author name.

\newcommand{\fix}{\marginpar{FIX}}
\newcommand{\new}{\marginpar{NEW}}
\newcommand{\X}{\mathcal{X}}


\nipsfinalcopy

\begin{document}

\maketitle

\begin{abstract}
Our project aims to model the functional connectivity of neuronal
microcircuits. On this scale, we are concerned with how the activity
of each individual neuron relates to the other neurons in the
population. Recent innovations, including the use of calcium indicator
dyes or multi-electrode arrays (MEA), allow researchers to collect
individual spiking activities of large groups of neighboring neurons,
supplying the data to address the fundamental problem of connectivity.
We will develop a model to infer a neural connectivity matrix given
spike train data that improves and builds on prior research efforts.
With a better understanding of the functional patterns of neural
activity at the cellular level, we can begin to decode the building
blocks of neural computation.
\end{abstract}

\section{Introduction}
\label{sec:introduction}

As we learn more and more about the workings of the neuron and of
specialized brain regions, the question increasingly becomes, how
do these pieces sum to a whole? How do the patterns of connectivity
give rise to vision, memory, motor function, and so on? Currently, a broad picture of the circuitry, or graphical connectivity,
of the brain does not exist, but several projects are underway to organize the solution of
this problem \citep{Marcus2011, Bohland2009}. Efforts to examine connectivity of the brain focus on scales ranging from brain regions each comprised of hundreds of
millions of cells down to microcircuits of only a few cells. Further,
some of these projects address structural connectivity and others
functional connectivity \citep{KnowlesBarley2011, Jain2010, Ropireddy2011, Chiang2011, bhattacharya2006}.

In this project, we will model the functional connectivity of
microcircuits: how electrical activity in one neuron influences
activities in other neurons. Importantly, functional connectivity
does not always imply anatomical connectivity; it only implies that
some set of neurons fire together in correlation.  Without anatomical
corroboration, these jointly firing neurons may have a common input
or be linked in a chain, rather than lie physically adjacent.
Providing experimental evidence of predicted connectivity between
pairs of neurons in the circuit is not in the scope of this project;
however, we intend to use real calcium imaging and multi-electrode array (MEA) data for our
analysis, and expect to see results consistent with the neural
connectivity literature.  Some measures of this consistency would
be that each neuron obeys Dale’s law, i.e. they are either
purely excitatory or inhibitory, and that connectivity is sparse.
With simulated data, we will also be able to see how sensitive
the connectivity inference is to various parameter structures and
decide whether these sensitivities or lack thereof are biologically
reasonable.

Several strategies have already been employed to infer the functional
connectivity of microcircuits from calcium imaging and MEA data
\citep{Gerwinn2010, takahashi2007, aguiar2009}. Of special interest to us and our approach
are two recent Bayesian approaches. In \citep{patnaik2011}, 
a pattern-growth algorithm is used to find repeated episodes of
activity. These patterns define mutual information between neurons
which they summarize in a dynamic Bayesian network. While their
methodology presents a contribution to the study of Bayesian networks,
one limitation of this work in inferring the connectivity of
microcircuits is that it only discovers relationships of excitation. In \citep{mishchencko2011}, network activity is modeled in terms of a collection of coupled hidden Markov chains, with each chain
corresponding to a single neuron in the network and the coupling
between the chains reflecting the network’s connectivity matrix.
To make computation feasible they used a blockwise-Gibbs sampling
method and took advantage of the parallel computing possibilities
when implementing their expectation-maximization algorithm. Although the work to date has done much to address the problem of
functional neural connectivity, there are still improvements to be
made to current models. For example, current models do not address
unobserved inputs to the system. By developing a new model, we can address these issues as well as explore potentially new
ways of inferring functional connectivity on the cellular scale. 

It is not yet clear how microcircuits code neural
computations, nor how they might generalize across brain regions
or individuals, but by uncovering the patterns of activity between
neurons we begin to decode these computational processes.

\section{Methods}
TBD.
\section{Experiments}
TBD.

\section{Conclusion}
We will develop a model to infer a neural connectivity matrix given spike train data (either continuous measurements, such as fluorescence levels, or discrete spikes) that improves and builds on prior research efforts. As ground truth data is limited in this domain, we will primarily measure performance using the same simulation processes and metrics used in the source papers for our model.  Specifically, we expect our model to accurately retrieve the connectivity parameters used to generate the simulated data.  

Further, since our team has access to limited lab data of spike trains, we will also compare our model against existing models on real data and expect our model to outperform on real data as well.  A potential metric for performance in this regime may come from the few experiments done, where dozens of cells are voltage clamped simultaneously, and functional connections are accurately teased out through direct measurements of post synaptic currents.  Certain graphical features have been inferred about the entire microcircuit from these limited electrophysiological data, such as the existence of scale-free, small world networks with certain two, three, and four cell motifs over-represented as compared to a randomly connected network \citep{song2005,perin2011}.  So in comparison to other models we may expect to validate these graphical features in the analysis of real data.

\begin{small}
\bibliographystyle{plainnat}
\bibliography{refs} 
\end{small}
\end{document}
